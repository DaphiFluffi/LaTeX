% Sandra Wimmer: s.wimmer.1@campus.tu-berlin.de
% Daphna Beljavskij: d.beljavskij@campus.tu-berlin.de

% Damit der Table of Contents erscheint muss man eventuell zwei mal kompilieren.
\documentclass[a4 paper]{article} 
    % Wir schreiben einen Artikel.

\usepackage[T1]{fontenc}
\usepackage[utf8]{inputenc}
\usepackage[ngerman]{babel} 

\begin{document}
    % Aufgabe 1
\title{Übungsaufgaben zur Vorlesung am 1. November 2017}
    % Setzt Dokumenttitel
\author{Sandra W.
            \thanks{
            Institut für Medien- und Kommunikationswissenschaften, 
            Große Ulrichstr. 51, 06108 Halle (Saale)}
        \and D. B. 
    % Durch \and können mehrere Autoren angegeben werden
            \thanks{ 
    % Durch \thanks{} können Ergänzungen zu den Autoren als Fußnote eingefügt werden.
            Institut für Theater-, Film-, und Medienwissenschaft,
            Goethe-Universität Frankfurt, 
            Institut für TFM, 60629 Frankfurt}
        \and Franz Dimpfelmoser 
            \thanks{
            Fraunhofer-Institut für Sichere Informationstechnologie,
            Rheinstr. 75, 64295 Darmstadt}
        \and Eva Schlotterbeck 
            \thanks{
            Institut für Germanistik,
            Universität Wien,
            Universitätsring 1, 1010 Wien, Österreich}
        } 
        
\maketitle
    % erstellt mit zuvor eingegebenen Parametern und Informationen das Titelblatt, die Formatierung ist festgelegt
    
\newpage

\setcounter{secnumdepth}{5}
    % legt die zu erfassenden Überschriften fest
\setcounter{tocdepth}{5}    
    % legt die aufzuführenden Überschriften fest
\tableofcontents 
    % erstellt das Inhaltverzeichnis nach zuvor festgelegten Parametern, Formatierung ist festgelegt

\newpage
    % erstellt einen Seitenumbruch

    % Aufgabe 2
\part{Newton} \label{Newton} 
    % Das \label{} erstellt eine Art Verknüpfung für die Überschrift, auf die mit den Befehlen \ref{} und \pageref{} zugegriffen werden kann.

    Um den grössten Physiker aller Zeiten in die Gemeinschaft der Physiker
    zurückzuführen, ist mir jeder Generalstab heilig. Es geht um die Freiheit unserer Wissenschaft und um nichts weiter. Wer diese Freiheit garantiert, ist gleichgültig. Ich diene jedem System, lässt mich das System in Ruhe. Ich weiss, man spricht heute von der Verantwortung der Physiker. Wir haben es auf einmal mit der Furcht zu tun und werden moralisch. Das ist Unsinn. Wir haben Pionierarbeit zu leisten und nichts ausserdem. Ob die Menschheit den Weg zu gehen versteht, den wir ihr bahnen, ist ihre Sache, nicht die unsrige. 
        \footnote{Noch nie ist eine Maus auf die Idee gekommen, ihre eigene Mausefalle zu konstruieren. - Albert Einstein}
    % \footnote{} fügt an verwendeter Stelle eine Fußnote ein, der Inhalt wird vorschriftsgemäß am Ende der Seite aufgeführt.

\section{Möbius} 
    % Dieser Befehl erzeugt einen neuen Abschnitt. In den geschweiften Klammern kann der Name des Abschnitts genannt werden.

    Auf der Universität winkte Ruhm, in der Industrie Geld. Beide Wege waren zu
    gefährlich. Ich hätte meine Arbeiten veröffentlichen müssen, der Umsturz unserer Wirtschaft und das Zusammenbrechen des wirtschaftlichen Gefüges wären die Folge gewesen.
        \footnote{Auch Kant hat schon einmal gesagt: Handle nur nach derjenigen Maime, von der du zugleich glauben kannst, dass sie ein allgemeines Gesetz werde.} 
    Die Verantwortung zwang mir einen anderen Weg auf. Ich liess meine akademische Karriere fahren, die Industrie fallen und überliess meine Familie ihrem Schicksal. Ich wählte die Narrenkappe. Ich gab vor, der König Salomo erscheine mir, und schon sperrte man mich in ein Irrenhaus. 

\subsection{Möbius} \label{Möbius} 
    % Die Subsection ist der "Unter-Abschnitt" der vorangehenden Section.

    Wir wissen einige genau erfassbare Gesetze, einige Grundbeziehungen zwischen unbegreiflichen Erscheinungen, das ist alles, der gewaltige Rest bleibt Geheimnis, dem Verstande unzugänglich. Wir haben das Ende unseres Weges erreicht. Aber die Menschheit ist noch nicht so weit.
        \footnote{Eine Handlung ist nach Jeremy Bentham nur dann moralisch vertretbar, wenn sie mehr Freude als Leid hervorbringt.}
    Wir haben uns vorgekämpft, nun folgt uns niemand nach, wir sind ins Leere gestossen. Unsere Wissenschaft ist schrecklich geworden, unsere Forschung gefährlich, unsere Erkenntnis tödlich.
    
\subsubsection{Frl Doktor} \label{Frl Doktor} 
    % Die Subsubsection ist der "Unter-unter-Abschnitt" der vorangehenden Subsection.

    Aber Möbius verriet ihn. Er versuchte zu verschweigen, was nicht verschwiegen werden konnte. Denn was ihm offenbart worden war, ist kein Geheimnis. Weil es denkbar ist. Alles Denkbare wird einmal gedacht. Jetzt oder in Zukunft. 

\paragraph{Möbius}
    % Der Paragraph ist eine Gliederungsebene unter der Subsubsection, verdeutlicht durch die unterschiedliche Formatierung.

    Was einmal gedacht wurde, kann nicht mehr zurückgenommen werden. 
    
\subparagraph{Friedrich Dürrenmatt}
    % Der Subparagraph ist erneut eine Ebene unter dem Paragraph, verdeutlicht durch eine wieder unterrschiedliche Formatierung.

    Die Physiker, laut Untertitel eine \\
    Komödie in zwei Akten, ist ein Drama des Schweizer Schriftstellers Friedrich Dürrenmatt. Es entstand im Jahr 1961 und wurde am 21. Februar 1962 unter der Regie von Kurt Horwitz im Schauspielhaus Zürich uraufgeführt. 1980 überarbeitete Dürrenmatt das Stück zu einer Endfassung für seine Werkausgabe.
 
\section{Aufgabe 4}

    Was der Absatz \ref{Newton} auf Seite \pageref{Newton} aussagt: Ich kann die Bewegung der Himmelskörper berechnen, aber nicht das Verhalten der  Menschen.
    
    Auf Seite \pageref{Möbius} in Kapitel \ref{Möbius} gesagt ist: Wenn die Weiber sich ihres Namens schämen sollten, so ist das schlimm genug, aber kein Grund, die Sprache zu vergewaltigen.
    
    Die Autorin des Absatzes \ref{Frl Doktor} der Seite \pageref{Frl Doktor} ist eine anerkannte Psychiaterin.

\section{Aufgabe 5}
  
    Wenn man nichts unterhalb der Ebene \verb+\section{}+ im Inhaltsverzeichnis aufgeführt haben möchte, so setzt man \verb+\setcounter{tocdepth}{1}+ vor den Befehl \verb+\tableofcontents+, da die Ebene der \verb+\section{}+ die 1 ist und durch diesen Befehl nur bis hin zu dieser Ebene das Inhaltsverzeichnis gebildet wird.

\section{Aufgabe 6}

    Wie man unten sieht, bewirkt der Befehl \verb+\begin{quote}+, dass das Blockzitat im Vergleich zum restlichen Text als Block eingerückt und es dadurch hervorgehoben wird. 
    Der Befehl \verb+\begin(quotation)+ führt zu einem ähnlichen Ergebnis, jedoch wird die erste Zeile im Gegensatz zum restlichen eingerückten Block zusätzlich eingerückt, was mehr Aufmerksamkeit auf das Zitat richtet. 

\subsection{Interview mit Friedrich Dürrenmatt}

    Haben Sie auch die Theaterstücke nur zum Geldverdienen geschrieben?
     
    \begin{quote} 
         DÜRRENMATT: "Natürlich spielte das auch bei den Stücken eine gewisse Rolle. Den "Besuch der alten Dame" habe ich in einer finanziellen Zwangslage geschrieben. Die Urfassung war völlig anders, aber die wäre für das Theater nicht brauchbar gewesen. Deshalb habe ich es dann umgeschrieben. Ursprünglich war das eine sehr groteske Geschichte. Da fährt ein Bauer, der in Amerika sehr viel Geld gemacht hat, mit einem riesigen Cadillac durch ein verschneites Bergtal in das Dorf, wo er herstammt, und will die Bewohner beschenken. " 
    \end{quote}

    Stimmt es, daß Sie als Kind den Wunsch hatten, Oberst zu werden?
    
    \begin{quotation}
         DÜRRENMATT: "Nein. Das hat ein Freund meines Vaters gesagt, als er meine Zeichnungen sah. Ich habe als Kind furchtbar gern Schlachten gezeichnet und bin mit einer Bohnenstange als Lanze herumgegangen. Meine Phantasie war ganz bildhaft. Eine typische Geschichte ist, wie mir meine Mutter einmal ein Buch gezeigt hat: Da war der Tod abgebildet in Gestalt eines Gerippes, und als ich sie fragte, was das sei, sagte sie, um mich nicht zu erschrecken, das sei Kaiser Wilhelm." 
    \end{quotation}
    
    Die in meinem Interview mit Friedrich Dürrenmatt gefallenen Äußerungen über die Kollegen Grass, Frisch und Hochhuth sind nach der Veröffentlichung in mehreren deutschsprachigen Zeitschriften nachgedruckt worden, was zur Folge hatte, daß der Schriftsteller von seinem Anwalt verbreiten ließ, das ganze Interview sei eine Fälschung. Später milderte er diese Behauptung ab. 
     
\end{document}