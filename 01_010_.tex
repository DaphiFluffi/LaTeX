%FACH: MintGrün
%FAKULTAET: 
%SEMESTER: 1
%ABSCHLUSS: 

%FACH: MintGrün
%FAKULTAET: 
%SEMESTER: 1
%ABSCHLUSS: 

\documentclass [12pt,a4paper]{letter} 
    %Die Dokumentenklasse ist "letter", was bedeutet, dass wir hier ein Briefdokument 	erstellen. Die zusätzlichen Befehle [12pt] und [a4paper] in den eckigen Klammern geben an, dass die Schriftgröße 12 und die Papierart A4 ist.
\usepackage[T1]{fontenc}
    %Die Anzahl der Schriftzeichen wurde eingstellt. Es sind 128 Schriftzeichen ohne Umlaute.
\usepackage[utf8]{inputenc}
    %Dieses Package gibt an, dass in diesem Dokument ASCII Zeichen, also lateinische Buchstaben verwendet werden.
\usepackage[ngerman]{babel} 
    %Dieses Package fügt die Worttrennung nach der neuen Rechtschreibung, eine deutsche Datumsausgabe und eine leichtere Eingabe von Umlauten ein.

\begin{document} 
    %\begin{document} grenzt die Präambel von dem Textkörper des Latexdokuments ab und beginnt das Dokument.

\address{Max Mustermann \\Musterstraße 45 \\ 10320 Berlin} 
    %\address beinhaltet die Anschrift des Absenders in formatierter Form.
\date{25. Oktober 2017} 
    %Der Befehl \date{} teilt dem Dokument ein Datum zu.
\signature{Max Mustermann \\ Director of Musterfirma} 
    %Dieser Befehlt fügt die Signatur des Absenders am Ende des Briefes ein. Er ist vorformatiert, sodass darüber Platz für eine Unterschrift auf dem Brief ist.

\begin{letter} 
    %Hier wird der Brief begonnen.
{Herr Petrosilius Zwackelmann \\
Am Eichholz 11 \\
21614 Buxtehude } 
    %Dies ist die Anschrift des Empfängers.

\opening{Sehr geehrter Herr Zwackelmann,} 
    %\opening{} beschreibt den Anfang des Brieftextes, indem der Befehl die anfängliche Grußformel einfügt.

    %Aufgabe 5: 
wenn man die Dokumentenklasse falsch schreibt, so erhält man keine pdf-Ausgabe, da die somit angeforderte Klasse für LaTeX unbekannt ist. \\
\\
Wenn man in der Klasse {letter} den Befehl \verb+\section{}+ benutzt, erhält man den Fehler 'undefined control sequence', weil der Befehl in der Dokumentenklasse {letter} nicht definiert und dadurch ungültig ist.\\
\\
Schreibt man \verb+\usepackage[ngerman]{babel}+ nach \verb+\begin{letter}+, so bekommt man 'can be used only in preamble' als Fehlermeldung, denn Packages lassen sich nur in der Präambel des Dokuments einfügen.\\
\\
    %Aufgabe 6: 
Wenn man statt \verb+\opening{Liebe Julia,}+ \verb+\opening{Liebe Julia},+ schreibt, rutscht das Komma in eine eigene Zeile, da es nicht im Opening enthalten und zwischen dem Komma und dem darauffolgenden Textstück eine Leerzeile vorhanden ist. 

\closing{Mit freundlichen Grüßen,} 
    %Der Befehl \closing{} fügt die abschließende Grußformel in den Brief ein.
\cc{Lukas Podolski, \ Manuel Neuer} 
    %Der Befehl \cc{} fügt die Angabe der weiteren Empfänger des Briefes ein.

\end{letter} 
    %\end{letter} beendet den Brief

\end{document} 
    %\end{document} beendet das Dokument.