\documentclass[10pt]{article}

\usepackage[utf8]{inputenc}
    % Package für das Inputencoding
\usepackage{tabularx}
    % Package für tabularx Tabellen
\usepackage{booktabs}
    % Package ermöglicht das einfügen von \toprule, \midrule, \bottomrule und \addlinespace
\usepackage{array}
    % Package erlaubt das Zusammenführen von Zellen
\usepackage{dcolumn}
    % Ermöglicht das Einbauen von Dezimalspalten
\usepackage{longtable}
    % Ermöglicht das Einbauen Seitenübergreifender Tabellen wie in Aufgabe 4
\usepackage[ngerman]{babel}
    % Erleichtert die Verwendung von deutschen Sonderzeichen


\title{Hausaufgaben zum 29.11.17}
\author{D. B. \and S. W.}
\date{27.11.2017}

\begin{document}
\maketitle
\section{1.Aufgabe}

\setlength{\tabcolsep}{6pt} 
    % Einstellen des Spaltenabstandes
\begin{tabular}{@{}l D{,}{.}{0}<{\qquad} D{,}{.}{-1} @{}} 
    
    % Das @{} entfernt den Überstand an den Rändern der Tabelle.
    % \qquad versetzt die rechtsbündige Spalte nach links
    % Die zwei dcolumn Befehle richten die zweite und die dritte Spalte der Vorgabe entsprechend aus.
    % Das Komma in der jeweils ersten Geschweiften Klammer bezeichnet das Seperatorzeichen in der TEX-Datei, in diesem Fall ein Komma.
    % Der Punkt in der jeweils zweiten geschweiften Klammer bezeichnet die Art und Weise, wie das Seperatorzeichen in der pdf ausgegben wird, in diesem Fall ein Punkt.
    % Die Angabe in der jeweils dritten geschweiften Klammer gibt die Maximle Anzahl an Dezimalstellen nach dem Komma an. Bei der mittleren Spalte wurde 0 gewählt, da wir keine Nachkommastellen angeben wollen. Das Minus bei -1 bedeutet, dass das Separatorzeichen innerhalb der Spalte zentriert wird.
    
\toprule
    % obere Tabellenlinie
Role & \multicolumn{1}{c}{No. of corrections} & \multicolumn{1}{r@{}}{\%~of~total~corrections}\\
    % Anpassung der Kopfspalten durch \multicolumn Befehle. In der Mittleren Spalte ist der Text zentriert. In der rechten Spalte ist der Text rechtsbündig und ohne Überschuss.
\midrule
    %Mittlere Linie
    Initiator & 1702 & 37 \\
    Supporter & 50 & 1 \\
    Advisor & 22 & 0,4 \\
    Suggester & 211 & 4,5 \\
    Provider & 2380 & 51 \\
    Mutator & 270 & 6 \\
\addlinespace
    % Fügt Abstand zwischen den oberen Tabelleneinträgen und dem "Total" ein
    Total & 4635 & 100 \\
\bottomrule
    %untere Linie

\end{tabular}

\section{2.Aufgabe}

\begin{tabularx}{\linewidth}{@{} X D{,}{.}{-5} D{,}{.}{2} @{}}
    % Benutzung des tabularx Packages, weil man hier die Zeilenbreite fest einstellen kann
    % \linewidth setzt die Tabelle auf Zeilenbreite
    % Verwendung von zwei Dezimalspalten zur Ausrichtung wie im Beispiel 
    %{-5} bedeutet, dass maximal 5 Dezimalstellen in diese Spalte eingefügt werden dürfen und die Kommata in der Spalte zentriert sind
    % Erklärungen zu @{}, {,}, {.} analog zu Aufgabe 1
\toprule
    % Linie über dem Spaltenkopf
Puddingsorte &\multicolumn{1}{c}{Messwert A} &\multicolumn{1}{r@{}}{Messwert B}\\
    % Ausrichtung von Messwert A und Messwert B mit Hilfe von multicolums befehlen. Messwert a wird zentriert und Messwert B ist rechtsbündig ohne Überstand.
\midrule
    % LInie unter em Spaltenkopf
Mandelpudding & 200,67678 & ,67\\
Schokoladenpudding & 10789777,22 & 10,1\\
Erdbeerpudding & ,29 & 3456835,35\\
Vanillepudding & 8 & 3\\
\bottomrule
    % Linie am Ende der Tabelle
\end{tabularx}

\section{3.Aufgabe }

\begin{tabular}[b]{cc} 
     % [b] heißt eigentlich "bottom" und bedeutet eine Ausrichtung oberhalb der Grundlinie. Das gilt auch für die zweite Tabelle, nur gibt es hier 4 Spalteneinträge.
    a & a \\
    b & b \\
    c & c \\
    
\end{tabular}
    \hfill
    %Fügt jeweils einen horzontal dehnbaren Abstand zwischen den Tabellen ein.
\begin{tabular}[b]{cc}
    a & a \\
    b & b \\
    c & c \\
    d & d \\
    
\end{tabular}
    \hfill
\begin{tabular}[c]{cc}
    % [c] bedeutet "entsprechend der Grundlinie Zentriert", diese Angabe kann aber auch weggelassen werden, da es die Standardformatierung ist.
    a & a \\
    b & b \\
    
\end{tabular}
    \hfill
\begin{tabular}[t]{cc}
    % [t] heißt eigentlich "top" und bedeutet eine Ausrichtung unterhalb der Grundlinie.
    a & a \\
    b & b \\
    c & c \\
    
\end{tabular}

\newpage 
    % Seitenumbruch für die 4. Aufgabe
\section{Aufgabe 4}

\scriptsize
    % Einstellen der Schriftgröße in der gesamen Tabelle auf scriptsize
\setlength{\LTleft}{-2cm}
    % Ausrichtung der Tabelle auf der Seite

\begin{longtable}{@{} l l l l l l @{}}
    % Tabellenpräambel: Definition von 6 linksbündigen Spalten ohne Überschuss auf beiden Seiten durch das Einfügen von @{}
    % Verwendung des longtable.sty packages, weil die Tabelle über mehrere Seiten geht
\toprule
    % Linie über dem Spaltenkopf
    Land & Hauptstadt & Fläche (qkm) & Einwohner (2012) & \multicolumn{1}{r}{BIP (Mrd. US\$, 2008)} & BIP/Kopf (2008) \\
    % rechtsbündige Ausrichtung des fünften Spaltenkopfes mit \multicolumn
\midrule
    % Linie unter dem Spaltenkopf
\endhead 
    % Definiert die Zeile, die am Anfang jeder Seite erscheint.
    % Ich habe in diesm Fall kein \endfirstehd verwendet,  weil sich der Spaltenkopf der ersten Seite nicht von den Anderen unterscheidet.
\bottomrule
    % Linie am Ende der Tabelle
\multicolumn{6}{c}{Fortsetzung folgt auf der nächsten Seite}\\
    % Einsetzen des Fortsetzungshinweises durch einen \multicolumn Befehl, der alle sechs Spalten umfasst. Der Text erscheint mittig unter der \bottomrule.
\endfoot
    % Beschreibt, dass der Fortsetzungshinweis am Ende jeder Seite ausgegeben werden soll.
\bottomrule
    % Setzt eine Linie unter die Tabelle ein.
\endlastfoot
    % Bescheibt, dass diese Linie nur am Ende der letzten Seite erscheinen soll.

    Albanien & Tirana & 28.748 & 3.162.000 & 12,96 & 4.088\\
    Andorra & Andorra la Vella & 468 & 78 & 3,50 & 41.722\\
    Belgien & Brüssel & 32.545 & 11.142.000 & 506,39 & 47.473\\
    Bosnien und Herzegowina & Sarajevo & 51.129 & 3.834.000 & 18,47 & 4.058\\
    Bulgarien & Sofia & 110.994 & 7.305.000 & 51,99 & 6.834\\
    Danemark & Kopenhagen & 43.098 & 5.590.000 & 342,93 & 62.624\\
    Deutschland & Berlin & 357.121 & 81.890.000 & 3.667,51 & 44.79\\
    Estland & Tallinn & 45.227 & 1.339.000 & 23,23 & 17.31\\
    Finnland & Helsinki & 338.144 & 5.414.000 & 273,98 & 51.43\\
    Frankreich & Paris & 543.965 & 65.697.000 & 2.865,73 & 44.637\\
    Griechenland & Athen & 131.957 & 11.280.000 & 357,55 & 32.093\\
    Irland & Dublin & 70.273 & 4.589.000 & 273,33 & 64.465\\
    Island & Reykjavík & 103 & 320 & 17,55 & 55.189\\
    Italien & Rom & 301.336 & 60.918.000 & 2.313,89 & 38.407\\
    Kasachstan & Astana & 146.700 & 480.000 & 132,23 & 8.061\\
    Kosovo & Priština & 10.887 & 1.806.000 & 3,8 & 2.111\\
    Kroatien & Zagreb & 56.542 & 4.267.000 & 69,33 & 15.444\\
    Lettland & Riga & 64.589 & 2.025.000 & 34,05 & 15.06\\
    Liechtenstein & Vaduz & 160 & 37 & 4,93 & 137.751\\
    Litauen & Vilnius & 65.301 & 2.986.000 & 47,30 & 14.102\\
    Luxemburg & Luxemburg & 2.586 & 531 & 54,97 & 111.501\\
    Malta & Valletta & 316 & 418 & 8,34 & 20.341\\
    Mazedonien & Skopje & 25.713 & 2.106.000 & 9,57 & 4.639\\
    Moldawien & Kischinau & 33.8 & 3.560.000 & 6,12 & 1.855\\
    Monaco & (Stadtstaat) & 2 & 38 & 3,67 & 111.212\\
    Montenegro & Podgorica & 13.812 & 621 & 4,82 & 7.173\\
    Niederlande & Amsterdam & 41.526 & 16.768.000 & 868,94 & 52.685\\
    Norwegen & Oslo & 323.759 & 5.019.000 & 456,23 & 94.535\\
    Osterreich & Wien & 83.879 & 8.462.000 & 415,32 & 49.579\\
    Polen & Warschau & 312.685 & 38.543.000 & 525,74 & 13.78\\
    Portugal & Lissabon & 92.345 & 10.524.000 & 244,49 & 23.026\\
    Rumänien & Bukarest & 238.391 & 21.327.000 & 199,67 & 9.287\\
    Russland & Moskau & 3.955.800 & 104.000.000 & 1.676,59 & 11.976\\
    San Marino & San Marino & 61 & 31 & 1,18 & 37.415\\
    Schweden & Stockholm & 449.964 & 9.517.000 & 484,55 & 52.271\\
    Schweiz & Bern & 41.285 & 8 139.600 & 492,60 & 68.433 \\
    Serbien & Belgrad & 88.361 & 7.224.000 & 50,06 & 5.384\\
    Slowakei & Bratislava & 49.034 & 5.410.000 & 95,40 & 17.489\\
    Slowenien & Ljubljana & 20.253 & 2.058.000 & 54,64 & 27.063\\
    Spanien & Madrid & 504.645 & 46.218.000 & 1.611,77 & 34.541\\
    Tschechien & Prag & 78.866 & 10.515.000 & 217,08 & 20.672\\
    Turkei & Ankara & 23.384 & 9.799.745 & 729,44 & 10.061\\
    Ukraine & Kiew & 603.7 & 45.593.000 & 179,73 & 3.908\\
    Ungarn & Budapest & 93.03 & 9.944.000 & 156,28 & 15.597\\
    Vatikanstadt & (Stadtstaat) & 0,44 & 800 & k.\,A. & k.\,A. \\
    Vereinigtes Konigreich & London & 242.91 & 63.228.000 & 2.674,09 & 43.756\\
    Weissrussland & Minsk & 207.595 & 9.464.000 & 60,29 & 6.354\\
    Albanien & Tirana & 28.748 & 3.162.000 & 12,96 & 4.088\\
    Andorra & Andorra la Vella & 468 & 78 & 3,50 & 41.722\\
    Belgien & Brüssel & 32.545 & 11.142.000 & 506,39 & 47.473\\
    Bosnien und Herzegowina & Sarajevo & 51.129 & 3.834.000 & 18,47 & 4.058\\
    Bulgarien & Sofia & 110.994 & 7.305.000 & 51,99 & 6.834\\
    Danemark & Kopenhagen & 43.098 & 5.590.000 & 342,93 & 62.624\\
    Deutschland & Berlin & 357.121 & 81.890.000 & 3.667,51 & 44.79\\
    Estland & Tallinn & 45.227 & 1.339.000 & 23,23 & 17.31\\
    Finnland & Helsinki & 338.144 & 5.414.000 & 273,98 & 51.43\\
    Frankreich & Paris & 543.965 & 65.697.000 & 2.865,73 & 44.637\\
    Griechenland & Athen & 131.957 & 11.280.000 & 357,55 & 32.093\\
    Irland & Dublin & 70.273 & 4.589.000 & 273,33 & 64.465\\
    Island & Reykjavík & 103 & 320 & 17,55 & 55.189\\
    Italien & Rom & 301.336 & 60.918.000 & 2.313,89 & 38.407\\
    Kasachstan & Astana & 146.700 & 480.000 & 132,23 & 8.061\\
    Kosovo & Priština & 10.887 & 1.806.000 & 3,8 & 2.111\\
    Kroatien & Zagreb & 56.542 & 4.267.000 & 69,33 & 15.444\\
    Lettland & Riga & 64.589 & 2.025.000 & 34,05 & 15.06\\
    Liechtenstein & Vaduz & 160 & 37 & 4,93 & 137.751\\
    Litauen & Vilnius & 65.301 & 2.986.000 & 47,30 & 14.102\\
    Luxemburg & Luxemburg & 2.586 & 531 & 54,97 & 111.501\\
    Malta & Valletta & 316 & 418 & 8,34 & 20.341\\
    Mazedonien & Skopje & 25.713 & 2.106.000 & 9,57 & 4.639\\
    Moldawien & Kischinau & 33.8 & 3.560.000 & 6,12 & 1.855\\
    Monaco & (Stadtstaat) & 2 & 38 & 3,67 & 111.212\\
    Montenegro & Podgorica & 13.812 & 621 & 4,82 & 7.173\\
    Niederlande & Amsterdam & 41.526 & 16.768.000 & 868,94 & 52.685\\
    Norwegen & Oslo & 323.759 & 5.019.000 & 456,23 & 94.535\\
    Österreich & Wien & 83.879 & 8.462.000 & 415,32 & 49.579\\
    Polen & Warschau & 312.685 & 38.543.000 & 525,74 & 13.78\\
    Portugal & Lissabon & 92.345 & 10.524.000 & 244,49 & 23.026\\
    Rumänien & Bukarest & 238.391 & 21.327.000 & 199,67 & 9.287\\
    Russland & Moskau & 3.955.800 & 104.000.000 & 1.676,59 & 11.976\\
    San Marino & San Marino & 61 & 31 & 1,18 & 37.415\\
    Schweden & Stockholm & 449.964 & 9.517.000 & 484,55 & 52.271\\
    Schweiz & Bern & 41.285 & 8 139.600 & 492,60 & 68.433 \\
    Serbien & Belgrad & 88.361 & 7.224.000 & 50,06 & 5.384\\
    Slowakei & Bratislava & 49.034 & 5.410.000 & 95,40 & 17.489\\
    Slowenien & Ljubljana & 20.253 & 2.058.000 & 54,64 & 27.063\\
    Spanien & Madrid & 504.645 & 46.218.000 & 1.611,77 & 34.541\\
    Tschechien & Prag & 78.866 & 10.515.000 & 217,08 & 20.672\\
    Turkei & Ankara & 23.384 & 9.799.745 & 729,44 & 10.061\\
    Ukraine & Kiew & 603.7 & 45.593.000 & 179,73 & 3.908\\
    Ungarn & Budapest & 93.03 & 9.944.000 & 156,28 & 15.597\\
    Vatikanstadt & (Stadtstaat) & 0,44 & 800 & k.\,A. & k.\,A. \\
    Vereinigtes Konigreich & London & 242.91 & 63.228.000 & 2.674,09 & 43.756\\
    Weissrussland & Minsk & 207.595 & 9.464.000 & 60,29 & 6.354\\
    Albanien & Tirana & 28.748 & 3.162.000 & 12,96 & 4.088\\
    Andorra & Andorra la Vella & 468 & 78 & 3,50 & 41.722\\
    Belgien & Brüssel & 32.545 & 11.142.000 & 506,39 & 47.473\\
    Bosnien und Herzegowina & Sarajevo & 51.129 & 3.834.000 & 18,47 & 4.058\\
    Bulgarien & Sofia & 110.994 & 7.305.000 & 51,99 & 6.834\\
    Danemark & Kopenhagen & 43.098 & 5.590.000 & 342,93 & 62.624\\
    Deutschland & Berlin & 357.121 & 81.890.000 & 3.667,51 & 44.79\\
    Estland & Tallinn & 45.227 & 1.339.000 & 23,23 & 17.31\\
    Finnland & Helsinki & 338.144 & 5.414.000 & 273,98 & 51.43\\
    Frankreich & Paris & 543.965 & 65.697.000 & 2.865,73 & 44.637\\
    Griechenland & Athen & 131.957 & 11.280.000 & 357,55 & 32.093\\
    Irland & Dublin & 70.273 & 4.589.000 & 273,33 & 64.465\\
    Island & Reykjavík & 103 & 320 & 17,55 & 55.189\\
    Italien & Rom & 301.336 & 60.918.000 & 2.313,89 & 38.407\\
    Kasachstan & Astana & 146.700 & 480.000 & 132,23 & 8.061\\
    Kosovo & Priština & 10.887 & 1.806.000 & 3,8 & 2.111\\
    Kroatien & Zagreb & 56.542 & 4.267.000 & 69,33 & 15.444\\
    Lettland & Riga & 64.589 & 2.025.000 & 34,05 & 15.06\\
    Liechtenstein & Vaduz & 160 & 37 & 4,93 & 137.751\\
    Litauen & Vilnius & 65.301 & 2.986.000 & 47,30 & 14.102\\
    Luxemburg & Luxemburg & 2.586 & 531 & 54,97 & 111.501\\
    Malta & Valletta & 316 & 418 & 8,34 & 20.341\\
    Mazedonien & Skopje & 25.713 & 2.106.000 & 9,57 & 4.639\\
    Moldawien & Kischinau & 33.8 & 3.560.000 & 6,12 & 1.855\\
    Monaco & (Stadtstaat) & 2 & 38 & 3,67 & 111.212\\
    Montenegro & Podgorica & 13.812 & 621 & 4,82 & 7.173\\
    Niederlande & Amsterdam & 41.526 & 16.768.000 & 868,94 & 52.685\\
    Norwegen & Oslo & 323.759 & 5.019.000 & 456,23 & 94.535\\
    Österreich & Wien & 83.879 & 8.462.000 & 415,32 & 49.579\\
    Polen & Warschau & 312.685 & 38.543.000 & 525,74 & 13.78\\
    Portugal & Lissabon & 92.345 & 10.524.000 & 244,49 & 23.026\\
    Rumänien & Bukarest & 238.391 & 21.327.000 & 199,67 & 9.287\\
    Russland & Moskau & 3.955.800 & 104.000.000 & 1.676,59 & 11.976\\
    San Marino & San Marino & 61 & 31 & 1,18 & 37.415\\
    Schweden & Stockholm & 449.964 & 9.517.000 & 484,55 & 52.271\\
    Schweiz & Bern & 41.285 & 8139.600 & 492,60 & 68.433 \\
    Serbien & Belgrad & 88.361 & 7.224.000 & 50,06 & 5.384\\
    Slowakei & Bratislava & 49.034 & 5.410.000 & 95,40 & 17.489\\
    Slowenien & Ljubljana & 20.253 & 2.058.000 & 54,64 & 27.063\\
    Spanien & Madrid & 504.645 & 46.218.000 & 1.611,77 & 34.541\\
    Tschechien & Prag & 78.866 & 10.515.000 & 217,08 & 20.672\\
    Turkei & Ankara & 23.384 & 9.799.745 & 729,44 & 10.061\\
    Ukraine & Kiew & 603.7 & 45.593.000 & 179,73 & 3.908\\
    Ungarn & Budapest & 93.03 & 9.944.000 & 156,28 & 15.597\\
    Vatikanstadt & (Stadtstaat) & 0,44 & 800 & k.\,A. & k.\,A. \\
    Vereinigtes Konigreich & London & 242.91 & 63.228.000 & 2.674,09 & 43.756\\
    Weissrussland & Minsk & 207.595 & 9.464.000 & 60,29 & 6.354\\


\end{longtable}

\end{document}
